% --- --- --- --- --- --- --- --- --- --- --- --- --- --- --- --- %

\documentclass[a4paper,12pt]{report}  % args : article, report, book 
\usepackage[utf8]{inputenc}  % encodage UTF-8 
\usepackage[french]{babel}  % choix de langue, important pour maketitle et tableofcontents
\usepackage{geometry}  % géométrie de la page 
\usepackage{hyperref}  % liens internes et externes 
\usepackage{graphicx}  % images 
\usepackage{longtable}  % tableaux 
\usepackage{fontspec}  % polices d'écriture du système 
\usepackage{tabularx}  % pour des tableaux larges 
\usepackage{lipsum}  % pour du texte 

% --- --- --- --- --- --- --- --- --- --- --- --- --- --- --- --- %

\geometry{margin = 2 cm}  % marge de 2 cm (possible d'utiliser "in" aussi)

% hyperref 
\hypersetup{  
	colorlinks=true,
	linkcolor=black,
	urlcolor=blue,
	citecolor=red
}

% fontspec
\setmainfont{Bricolage Grotesque}
%\setmonofont[size=6pt]{Comic Code Ligatures} 


% --- --- --- --- --- --- --- --- --- --- --- --- --- --- --- --- %

\title{Cahier des charges fonctionnel}
\author{Groupe LRSVZZ}
\date{\today}

\begin{document}
	
	\maketitle
%	\pagebreak
	
	% --- --- --- --- %
	
	\section*{Auteurs}
	
	\renewcommand{\arraystretch}{1.5}  % augmenter de 50 % les lignes 
	\begin{table}[h]
%		\centering
%		\caption{Auteurs}
%		\begin{tabular}{|c|c|c|}
		\begin{tabularx}{\textwidth}{|X|X|X|}  % on met X pour des colonnes de même longueur 
			\hline
			\textbf{Nom} & \textbf{Rôle} & \textbf{Département} \\
			\hline
			VARTANIAN Djivan & Responsable & Informatique \\
			\hline
			Row 2, Col 1 & Row 2, Col 2 & Row 2, Col 3 \\
			\hline
			Row 3, Col 1 & Row 3, Col 2 & Row 3, Col 3 \\
			\hline
%		\end{tabular}
		\end{tabularx}
	\end{table}
	
	\section*{Historique des documents}
	\begin{table}[h]
%		\centering
		\begin{tabular}{|c|c|c|c|} 
			\hline
			\textbf{Date} & \textbf{Version} & \textbf{Description du document} & \textbf{Auteur du document} \\
			\hline
			& & & \\
			\hline
			& & & \\
			\hline
			& & & \\
			\hline
		\end{tabular}
	\end{table}
	
	\section*{Approbations}
	\begin{table}[h]
		\begin{tabular}{|c|c|c|c|} 
			\hline
			\textbf{Date d'approbation} & \textbf{Version approuvée} & \textbf{Rôle de l'approbateur} & \textbf{Approbateur} \\
			\hline
			& & & \\
			\hline
			& & & \\
			\hline
			& & & \\
			\hline
		\end{tabular}
	\end{table}
	
	% --- --- --- --- %
		
	\tableofcontents
	
	\pagebreak
	% --- --- --- --- %
	
	\section{Introduction}
%	Identifiez le besoin ou le problème de l'entreprise.
	Introduction.
	\lipsum[1] 
	
	\begin{figure}[h]
		\centering
		\includegraphics[width=\textwidth]{./attachements/Untitled-2025-06-11-1251_B.pdf}
		\caption{Description. }
		\label{fig:your_label}
	\end{figure}
	
	\subsection{Objet du document}
%	Décrivez le cahier des charges fonctionnel et son objectif.
%	Le cahier des charges fonctionnel décrit comment la solution système fonctionnera et son comportement requis.
	\lipsum[2] 
	
	\subsection{Portée du projet}
%	Décrire la portée du projet et la solution proposée.
	\lipsum[3]
	
	\subsection{Champ d'application du document}
%	Décrire le champ d'application spécifique du document.
	\lipsum[4]
	
	\subsection{Documents connexes}
%	Ajouter toute documentation pertinente.
	\begin{longtable}{|l|l|l|}
		\hline
		\textbf{Composant} & \textbf{Nom (lien)} & \textbf{Description} \\
		\hline
	\end{longtable}
	
	\subsection{Termes/acronymes et définitions}
%	Indiquez les termes et définitions.
	\lipsum[5]
	\begin{longtable}{|l|l|l|}
		\hline
		\textbf{Terme/Acronyme} & \textbf{Définition} & \textbf{Description} \\
		\hline
	\end{longtable}
	
	\subsection{Risques et hypothèses}
%	Énumérer les risques et hypothèses affectant la conception.
	\lipsum[6]
	
	\pagebreak
	% --- --- --- --- %
	
	\section{Aperçu du système/de la solution}
%	Brève description du logiciel et de la solution.
	\lipsum[7]
	
	\subsection{Diagrammes}
%	Fournir des représentations graphiques appropriées.
	\lipsum[8]
	
	\subsection{Acteurs du système}
	\subsubsection{Rôles et responsabilités des utilisateurs / exigences en matière d'autorité}
	\lipsum[9]
	\begin{longtable}{|l|l|l|l|l|}
		\hline
		\textbf{Utilisateur/Rôle} & \textbf{Exemple} & \textbf{Fréquence d'utilisation} & \textbf{Sécurité/accès} & \textbf{Notes} \\
		\hline
	\end{longtable}
	
	\subsection{Dépendances et impacts des changements}
	\subsubsection{Dépendances du système}
%	Liste des dépendances avec d'autres systèmes.
	\lipsum[10]
	
	\subsubsection{Impacts du changement}
%	Identifier les systèmes impactés par la solution.
	\lipsum[11]

	\pagebreak
	% --- --- --- --- %
	
	\section{Spécifications fonctionnelles}
%	Décrire les spécifications du système global.
	\lipsum[12]
	
	\subsection{Titre}
	\subsubsection{Objectif / Description}
%	Description de haut niveau des spécifications.
	\lipsum[13]
	
	\subsubsection{Cas d'utilisation}
%	Mapper l'exigence fonctionnelle à des cas d'utilisation.
	\lipsum[14]
	
	\subsubsection{Maquette}
%	Fournir la maquette de la fonctionnalité.
	\lipsum[15]
	
	\subsubsection{Exigences fonctionnelles}
%	Détails au niveau de la page.
	\lipsum[16]
	\begin{longtable}{|l|l|l|}
		\hline
		\textbf{ID du spécimen} & \textbf{Description} & \textbf{Règles de gestion} \\
		\hline
	\end{longtable}
	
	\subsubsection{Spécifications au niveau du terrain}
%	Éléments de données de champ liés à l'exigence fonctionnelle.
	\lipsum[17]
	
	\pagebreak
	% --- --- --- --- %
	
	\section{Configurations du système}
%	Vue d'ensemble des étapes de configuration nécessaires.
	\lipsum[18]
	
	\section{Autres exigences du système/exigences non fonctionnelles}
%	Détails supplémentaires sur les aspects liés à la qualité.
	\lipsum[19]
	
	\section{Exigences en matière de rapports}
%	Définir les besoins en matière de rapports.
	\lipsum[20]
	
	\section{Exigences d'intégration}
%	Identifier les besoins d'intégration et les interfaces nécessaires.
	\lipsum[21]
	
	\subsection{Traitement des exceptions/rapports d'erreurs}
%	Expliquer les conditions d'erreur/exceptions.
	\lipsum[22]
	
	\begin{longtable}{|l|l|l|l|}
		\hline
		\textbf{ID de l'exception} & \textbf{Erreur} & \textbf{Cause} & \textbf{Stratégie de solution} \\
		\hline
	\end{longtable}
	
	\section{Exigences en matière de migration/conversion des données}
%	Expliquer le plan de conversion des données.
	\lipsum[23]
	
	\subsection{Stratégie de conversion des données}
%	Stratégie globale pour la conversion des données.
	\lipsum[24]
	
	\subsection{Préparation à la conversion des données}
%	Détails sur les conditions préalables à la conversion.
	\lipsum[25]
	
	\subsection{Spécifications pour la conversion des données}
	\lipsum[26]
	\begin{longtable}{|l|l|l|l|l|}
		\hline
		\textbf{Source} & \textbf{Source Élément de données} & \textbf{Cible} & \textbf{Élément de données cible} & \textbf{Règles de conversion} \\
		\hline
	\end{longtable}
	
	\section{Références}
%	Listez toutes les références à des documents externes.
	\lipsum[27]
	
	\section{Questions ouvertes}
	\lipsum[28]
	
	\begin{longtable}{|l|l|l|l|l|l|l|l|}
		\hline
		\textbf{ID de la question} & \textbf{Enjeu} & \textbf{Élevé par} & \textbf{Décision} & \textbf{Résolu par} & \textbf{Statut} \\
		\hline
	\end{longtable}
	
	\section{Annexe}
%	Inclure toute information supplémentaire ou documents annexes.
	\lipsum[29]
	
\end{document}
